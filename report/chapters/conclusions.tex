\section{CONCLUSIONS}

In this project, we aimed to investigate the correlation between GDP, interest rates, and inflation by examining historical economic data from leading economies. Our analysis wanted to validate the current economic theory that hypothesizes an inverse relationship between GDP growth and interest rates and a correlation between GDP and inflation.

\subsection*{Correlation Analysis}
Our findings indicate a significant correlation between GDP and inflation, supporting the hypothesis that higher inflation often helps GDP growth. This aligns with the economic theory that as economies expand, the increased demand for goods and services tends to drive prices up.

The relationship between GDP and interest rates appears to be more complex. While an inverse correlation was observed in some cases, the strength and consistency of this relationship varied across different countries and time periods. This suggests that while interest rates are a critical tool for managing economic growth, their impact on GDP is influenced by a multitude of factors.

\subsection*{Model Predictions}
Utilizing Hidden Markov Models (HMM) for predicting economic indicators yielded good results. The models performed reasonably well on historical data, with predictions closely matching the actual economic trends. The confidence intervals were sound, and we can safely say that our prediction, although simplified, worked reasonably well.

Using the same model, we managed to find an higher prediction for the country economy compared to the result caused by the pandemic and the energy crisis that followed. These events have disrupted the economies and the lives of billions of people around the world, and with a more fine tuned model that uses more steps than just positive / negative changes (as explained in the introduction of the Hidden Markov Model chapter \ref{sec:hmm}) we could have a more precise prediction of how much growth did the recent events slowed down.

\subsection*{Policy Implications}
By finding correlations between the three variables studied, in particular between Inflation and Interest Rate, we can support the current economic theory and highlight the importance of the monetary policies the Central Banks apply in order to stabilize the economies of their countries.

It is important for policy makers to accurately calculate how much to fine tune, to avoid going too far and damage what they are trying to protect. The more precise analysis we described above, together with more accurate and frequent data, could be a valid assistance in choosing the best course of action.
