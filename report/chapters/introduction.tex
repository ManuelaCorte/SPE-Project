\section{INTRODUCTION}
In recent years, crisis like the COVID-19 pandemic and the war in Ukraine have severely limited the GDP growth of all nations, slowing down progress and making life for families and industries more difficult than ever.

Three main factors have increased inflation and the price of products: first, the global traffic of goods stopped because of the lock downs; then, the Russian sanctions increased the energy cost to produce them; finally, the recovery of the free trade increased demand.

To counter that, the current economic theory states that in order to decrease inflation central banks need to increase interest rates, in order to keep around 3\% the increase of prices. Too high money devaluation makes investments and currency useless, while too low devaluation creates economic stagnation. This policies are made to keep stable growth, measured in Gross Domestic Product (GDP).

But is this claim true? Is there really a correlation between GDP and inflation, and an inverse correlation between GDP and interest rate? In this short paper, our aim is to first find this correlation by looking at the economic figures history of some of the best economies in the world; then, if it exists, use it to make a prediction of the last years without the crises, to see how much they influenced negatively the well being of the world.

All code produced, as well as a copy of the paper, can be found on GitHub at \href{https://github.com/ManuelaCorte/SPE-Project}{https://github.com/ManuelaCorte/SPE-Project}.
