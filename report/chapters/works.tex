\section{RELATED WORKS}
\label{sec:related_works}
The task of understanding and predicting the relationships between economic indicators has been the subject of many studies in the past. In this section, we will review some of the most relevant works in this field.

\subsection*{Economic Indicators}
\label{subsec:economic_indicators}
The Gross Domestic Product (GDP) is a measure of the economic performance of a country. It represents the total value of all goods and services produced in a country over a specific period of time. The GDP is considered to be one of the most important economic indicators, as it provides a comprehensive view of the overall health of an economy.

The other macroeconomic indicators that we will consider in our work are:
\begin{itemize}
  \item Interest Rate: The rate of interest measures the percentage reward a lender receives for deferring the consumption of resources until a future date. Correspondingly, it measures the price a borrower pays to have resources now.
  \item Consumer Price Index (CPI): The CPI is a measure of the prices of a representative basket of goods and services purchased by a typical household. It is considered to be a key indicator of inflation, as it reflects the cost of living for citizens.
\end{itemize}

Different approaches have been used to model the relationships between these economic indicators. For example, in \cite{gdp_early} a regression model is used to analyze the relationships between GDP, interest rates, exchange rates, and other market indicators such as Sensex. The authors find that inflation is highly correlated with GDP but does not have a significant impact on GDP growth while market indicators and exchange rates have a significant correlation with GDP.

Another study \cite{gdp_regr} uses a Partial Least Squares (PLS) model and path analysis with the intent of explaining direct and indirect relationships between GDP, interest rate, exchange rate, and inflation. Similarly to \cite{gdp_early}, the authors find that inflation has a limited impact on GDP, while the exchange rate has a positive correlation with GDP. As for the interest rate, it has a negative impact on GDP. It is also interesting to consider the relation between interest rates and inflation which are negatively correlated. This is to be expected as increasing interest rates is used as a tool to control inflation as describe in \cite{imf}.
